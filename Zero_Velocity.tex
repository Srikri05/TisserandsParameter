\subsubsection{Zero-velocity surface}
Zero-velocity surface lays a boundary, dynamically, for mass $m_3$; inside the region, we observe the exclusion of the mass from any type of movement. 
\begin{equation}
    V(x, y, z)= J
\end{equation}
where,
\begin{equation}
    V(x, y, z) = 2\frac{\mu_1}{\rho_1} + 2\frac{\mu_2}{\rho_2} + x^2 + y^2
\end{equation}
If the mass $m_3$ has the Jacobi Integral $J$, and it lies on curve $V(x, y, z)$, then it must possess a zero velocity surface. 
This region grows with increase in the size of the Jacobi Integral $J$. The Jacobi is greatest for $J>J_1$ and subsequently reduces with higher $(J_i)$ Jacobi integrals. Here, $i$ represents the corresponding Lagrangian point.